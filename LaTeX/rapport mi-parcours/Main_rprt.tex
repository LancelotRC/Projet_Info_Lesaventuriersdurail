%! Author = lalol
%! Date = 24/03/2025


\documentclass[a4paper,12pt]{report}
\usepackage[utf8]{inputenc}
\usepackage[french]{babel}
\usepackage{geometry}
\geometry{margin=2.5cm}
\usepackage{graphicx}
\usepackage{hyperref}

\title{Rapport de mi-parcours \\ \textbf{Projet Info}}
\author{Zachary BARBON-EVIS \\ Lancelot RAMIS \\}
\date{\today}

\begin{document}

\maketitle

\tableofcontents
\newpage

\chapter{Introduction}

\section{Contexte et enjeux}
Présentation détaillée du contexte du projet, motivations, enjeux technologiques ou scientifiques.

\section{Objectifs du projet}
Énoncer clairement les objectifs visés par le projet.

\chapter{État actuel des travaux}

\section{Travaux réalisés}
Description précise des tâches accomplies jusqu’à présent, incluant éventuellement des illustrations,
captures d'écran ou résultats intermédiaires.

\section{Difficultés rencontrées}
Identifier et décrire les difficultés techniques ou organisationnelles apparues.

\chapter{Analyse intermédiaire}

\section{Résultats préliminaires}
Présentation et interprétation des premiers résultats ou des premières expérimentations.

\section{Évaluation du planning initial}
Analyse du respect du calendrier initialement fixé.

\chapter{Prochaines étapes}

\section{Planification à venir}
Décrire précisément les étapes suivantes avec des jalons et dates estimées.

\section{Axes d'amélioration}
Décrire les améliorations à apporter à l'organisation, la méthodologie ou aux aspects techniques.

\chapter*{Conclusion intermédiaire}
Synthèse des enseignements tirés à ce stade et ouverture sur les perspectives.

\appendix
\chapter{Annexes}
Ajouter ici des éléments complémentaires comme du code, schémas techniques, ou données expérimentales.

\end{document}
